%% Не менять - Do not modify
%%\thispagestyle{empty}%
%\setcounter{sumPageFirst}{\value{page}} %сохранили номер первой страницы Реферата
\ifnumequal{\value{sumPrint}}{1}{% если двухсторонняя печать Задания, то...
	\newgeometry{twoside,top=2cm,bottom=2cm,left=3cm,right=1cm,headsep=0cm,footskip=0cm}
	\savegeometry{MyTask} %save settings
	\makeatletter % задаём оформление второй страницы ВКР как нечетной, а третьей - как чётной
	\let\tmp\oddsidemargin
	\let\oddsidemargin\evensidemargin
	\let\evensidemargin\tmp
	\reversemarginpar
	\makeatother
}{} % 
\pagestyle{empty} % удаляем номер страницы на втором/третьем листе 
\chapter*[Count-me]{  \centerline{Реферат}  } % * - не нумеруем
\thispagestyle{empty}% удаляем параметры страницы
%\setcounter{sumPageFirst}{\value{page}}
%sumPageFirst \arabic{sumPageFirst}
%
%
%% Возможность проверить другие значения счетчиков - debugging
%\ref*{TotPages}~с.,
%\formbytotal{mytotalfigures}{рисун}{ок}{ка}{ков},
%\formbytotal{mytotaltables}{таблиц}{у}{ы}{},
%There are \TotalValue{mytotalfigures} figures in this document
%There are \TotalValue{mytotalfiguresInApp} figuresINAPP in this document
%There are \TotalValue{mytotaltables} tables in this document
%There are \TotalValue{mytotaltablesInApp} figuresINAPP in this document
%There are \TotalValue{myappendices} appendix chapters in this document
%\total{citenum}~библ. наименований.



%% Для того, чтобы значения счетчиков корректно отобразились, необходимо скомпилировать файл 2-3 раза
Пояснительная записка  72 cтр.,  %\total{mypages}
\formbytotal{myfigures}{рисун}{ок}{ка}{ков},
\formbytotal{mytables}{таблиц}{у}{ы}{},
%\formbytotal{myappendices}{приложен}{ие}{ия}{ий}.  

Ключевые слова: обучение с подкреплением, оптимальное управление, динамическое программирование, приближенное динамическое программирование, адаптивное управление, оптимально-адаптивный регулятор, нейродинамическое программирование


Тема выпускной квалификационной работы: Применение методов обучения с подкреплением в задачах управления 

Объект исследования --- методы обучения с подкрепленим. Предмет исследования --- определить связь методов обучения с подкреплением с адаптивными и оптимальными подходами в теории автоматического управления.

Цель исследования --- проанализировать современные методы искусственного интеллекта, а именно метод обучения с подкреплением, на предмет применимости для решения задач автоматизированного управления сложных систем.

В данной работе изложена сущность современного метода машинного обучения -- обучения с подкреплением для задачи автоматизированного управления сложных систем. Представленны общая терминология и методы обучения с подкреплением для задач управления. Показаны методы обучения с подкреплением с позиции решения задачи приближенного динамического программирования.  Показано сходство косвенных адаптивных регуляторов с методами обучения с подкреплением на базе моделей и сходство структуры прямых адаптивных регуляторов с безмодельными методами обучения с подкреплением. Реализованы регуляторы на базе алгоритмов глубокого обучения с подкреплением для перевернутого маятника, роста опухоли и производства пенициллина.


\newpage
\printTheAbstract % не удалять
%\total{mypages}~p., 
%\total{myfigures}~figures, 
%\total{mytables}~tables,
%\total{myappendices}~appendices.

Keywords: reinforcement learning, optimal control, dynamic programming, approximate dynamic programming, adaptive control, optimal-adaptive controller, neurodynamic programming


The subject of the graduate qualification work is: Application of reinforcement learning for control systems problems

The object of the study is reinforcement learning methods. The subject of the research is to determine the relationship between reinforcement learning methods and adaptive and optimal approaches in the theory of automatic control.

The purpose of the study is to analyze modern methods of artificial intelligence, namely, the reinforcement learning method, for applicability for solving problems of automated control of complex systems.

This paper outlines the essence of the modern method of machine learning -- reinforcement learning for the problem of automated control of complex systems. General terminology and reinforcement learning methods for management tasks are presented. Reinforcement learning methods from the position of solving the approximate dynamic programming problem are shown. The similarity of indirect adaptive controllers with model-based reinforcement learning methods and the similarity of the structure of direct adaptive controllers with modelless reinforcement learning methods are shown. Implemented deep-learning reinforcement-based controllers for inverted pendulum, tumor growth, and penicillin production.
	


%% Не менять - Do not modify
\thispagestyle{empty}
%\setcounter{sumPageLast}{\value{page}} %сохранили номер последней страницы Задания
%\setcounter{sumPages}{\value{sumPageLast}-\value{sumPageFirst}}
%sumPageLast \arabic{sumPageLast}
%
%sumPages \arabic{sumPages}
%\restoregeometry % восстанавливаем настройки страницы
%\setcounter{sumPageLast}{\value{page}} %сохранили номер последней страницы Задания
\setcounter{sumPages}{\value{sumPageLast}-\value{sumPageFirst}}
\arabic{sumPageLast}
\arabic{sumPages}
\restoregeometry % восстанавливаем настройки страницы
\pagestyle{plain} % удаляем номер страницы на первой/второй странице Задания
%% Обязательно закомментировать, если получается один лист в задании:
\ifnumequal{\value{sumPages}}{0}{% Если 1 страница в Задании, то ничего не делать.
}{% Иначе
	\setcounter{page}{\value{page}-\value{sumPages}} 	% вычесть значение sumPages при печати более 1 страницы страниц
}%	% настройки - конец