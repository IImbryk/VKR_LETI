% !TeX spellcheck = russian_english
\chapter{Разработка бизнес-плана по коммерциализации проекта} \label{ch4}
\section{Описание проекта}

\textit{Резюме.} Бизнес-план посвящен оценки рентабельности оказания услуги по разработке алгоритмов управления сложных технических систем с применением методов обучения с подкреплением.
Потенциальные заказчики услуги -- промышленные предприятия и компании в области разработки и проектирования автоматизированных систем управления технологическим процессом (АСУ ТП). Для существующих подходов в управлении отдельными техническими системами на рынке АСУ ТП характерна высокая стоимость и длительный цикл разработки. Поэтому предлагается использовать методы обучения с подкреплением, с целью уменьшить время разработки, без потери качеств управления.

В первый квартал планируется реализация и внедрение двух алгоритмов автоматизации, стоимость каждого -- 700 000 рублей. В последующие кварталы планируется объем -- 3 единиц услуг, по 500 000 рублей каждая.
В связи со спецификой технологии оказываемых услуг, необходимо проводить активную кампанию по продвижению услуги на рынке, что подразумевает затраты на рекламу в размере 90 600 рублей без НДС в первый год оказания услуг.
Бизнес-план составлен на прогнозный период 1 год. Срок организации бизнеса составляет 1 месяц, предполагается, что со 2-го месяца проект начнет приносить доход. В качестве расчетов принят календарный год с мая по апрель. Для расчета инвестиционной привлекательности проекта использовались следующие допущения:
\begin{itemize}
\item организация использует общую систему налогообложения;
\item налог на добавленную стоимость (НДС) – 20\%;
\item отчисления работодателя с доходов сотрудника – 30\%;
\item налог на прибыль – 20\%.
\end{itemize}

Полученные показатели инвестиционной привлекательности проекта с учетом ставки дисконтирования (20 \%): 
\begin{itemize}
\item сумма инвестиций (I) составляет 500 000 рублей;
\item чистая текущая стоимость проекта (NPV) составляет 1 830 000 рублей;
\itemсрок окупаемости проекта составит 1 квартал.
\end{itemize}
Полученные показатели указывает на экономическую целесообразность осуществления проекта с учетом ограничений и допущений в бизнес-плане.


\textit{Описание продукции}.Описание услуги: Разработка алгоритмов интеллектуальных систем управления сложных технологических объектов с применением методов обучения с подкреплением.

Цифровизация промышленности и, как следствие, автоматизация производственных процессов - современный тренд, движение в сторону которого, обеспечивает предприятиям высокую гибкость в формировании бизнес-моделей и широкий охват потенциальной клиентской базы посредством интеграции киберфизических систем и интернета вещей в производственный процесс. В основе внедрения новых технологий лежит стремление к комплексному повышению эффективности и созданию условий для успешной работы предприятия. Промышленные компании сталкиваются с задачами снижения издержек, сокращения сроков вывода новой продукции на рынок, улучшения эффективности всех процессов, поскольку требования со стороны потребителей данной услуги каждый год растут.
Поэтому, со стороны производства возникает запрос на автоматизацию производственных процессов, с применением современных технологий, которые смогут обеспечить адаптивность, отказоустойчивость и оптимальность процессов. 
Компания Forrester указывает в своем отчете, что 20\% компаний создадут инновационные цифровые подразделения в ближайшие годы. То есть ряд промышленных предприятий будет пробовать реализовывать и внедрять новые технологии автоматизации самостоятельно. Существует ряд компаний, которые уже давно на рынке и готовы внедрять надежные системы автоматизации, с помощью традиционных подходов. 
Компании, которые в настоящее время поставляют системы автоматического управления, при синтезе регуляторов, по большей части, используют классические подходы теории автоматического управления. Тогда как наука шагнула вперед и теперь существуют лучшие подходы в разработке алгоритмов управления технических систем.

Обращаясь к традиционным подходам в управлении сложных технических систем при создании систем автоматического управления необходимо иметь точную математическую модель объекта управления (ОУ). Во многих реальных задачах построение такой модели либо невозможно, либо требует проведения трудоёмких исследований. При этом параметры ОУ могут изменяться в широких пределах в процессе функционирования системы, либо иметь большой разброс значений от образца к образцу. В таких случаях регуляторы с постоянными настройками не всегда могут обеспечить требуемое качество работы системы. 

Если обратиться к области машинного обучения, то существует ряд методов, которые могут помочь в задаче управления технических систем. Обучение с подкреплением -- группа методов, при которых алгоритм учится выполнять задачу, многократно взаимодействуя с симулятором динамической системой. Это происходит без прямого вмешательства человека и без необходимости программировать алгоритм для выполнения конкретной задачи.

В рамках данного проекта предлагается услуга по разработке алгоритмов управления на основе методов обучения с подкреплением и внедрение их в микроконтроллеры. 

Технические характеристики предоставляемой услуги будет:
\begin{enumerate}[1.]
	\item Алгоритм регулирования на высокоуровневом языке программирования;
	\item Подключения к оборудованию по промышленным протоколам передачи данных;
	\item Визуализация работы алгоритма, создание автоматизированных отчетов;
	\item Алгоритм конвертации кода на промышленный язык программирования.
\end{enumerate}

Самая большая проблема рынка автоматизации производства -- это первоначальная стоимость инвестиций, необходимых для проектирования, выполнения и установки автоматизированной системы, а так же малое колличество подготовленных специалистов на стыке разных дисциплин. Предполагается, что предоставляя услугу по разработке алгоритмов управления с использованием обучения с подкреплением, можно в разы ускорить и удешевить процесс разработки регуляторов, увечив качество управления за счет адаптации к системе.
Так же сложности возникают, если на предприятии уже имеется оборудование автоматизации отдельного производителя, предлагаемая услуга не привязана к модели контроллеров, датчиков или исполнительных механизмов. То есть является универсальным решением для АСУ ТП.


\textit{Анализ рынка сбыта.} По данным аналитиков компании Fortune Business Insights, к 2026 году рынок промышленной автоматизации вырастет до суммы в \$310 миллиардов, при совокупном среднегодовом темп роста около 8,5\%.
Оценивая основные экономические характеристики отрасли, отметим, что объем российского рынка АСУ ТП в 2020 году составил 58,7 млрд рублей, подсчитали аналитики компании J’son \& Partners Consulting \cite{eco_1}. Рынок находится на растущем этапе развития, следовательно и сегменту по разработке программного обеспечения присущ рост. На рынке заметна нехватка мощностей, что вызвано стимуляцией цифровизации промышленности со стороны государства. 
Ожидается, что усиленное внимание к повышению производительности и стремление к устранению опасных ручных процессов с привлечением человека -- станет основными движущими силами в ближайшие несколько лет. Отдельно стоит отметить, что пандемия коронавируса уже стала одним из факторов роста рынка автоматизации.

На рынке АСУ ТП можно выделить минимум 2 сегмента по цене. Так как сложно найти информацию по средней цене, затрачиваемой предприятиями на автоматизацию, стоит отметить, что рассматриваемая услуга будет направлена на предприятия среднего и ниже среднего ценового сегмента. Например, такие компании как ПАО «СИБУР Холдинг», ПАО «Газпром» могут позволить себе полноценный комплекс автоматизации от ключевых игроков рынка (Siemens, Schneider Electric и т.д.), которые помимо классических решений регулирования, так же предлагают современные адаптивно-оптимальные подходы. Но в данном случае существует привязка к оборудованию. Данных проект направлен на заказчиков, которые имеют оборудования автоматизации от разных производителей и при этом доход таких компаний относится к средним показателям по рынку автоматизации.

Потенциальные заказчики услуги:
\begin{itemize}
\item Промышленные предприятия среднего ценового сегмента;
\item Компании в области разработки и проектирования АСУ ТП среднего ценового сегмента.
\end{itemize}

Высокие затраты на установку и техническое обслуживание, а также отсутствие квалифицированных специалистов являются одними из ограничений в этой отрасли.
Услуга охватывает все сегменты предприятий по отраслям: нефтегазовая, металлургическая и горнодобывающая, фармацевтическая, целлюлозно-бумажная, автомобильная, химическая и т.д. Для любых производственных компаний, занимающихся данной деятельностью, важнейшим фактором выживаемости в конкурентной борьбе является постоянное обновление проектов в соответствии с современными стандартами и технологиями.

Лидеры рынка АСУ ТП неизменны на протяжении многих лет, их решения проверены временем и используются на многих предприятиях. Стабильность позиций во многом связана с тем, что производители предлагают законченную инфраструктуру, и с точки зрения эксплуатации клиентам выгодно иметь оборудование одного вендора, так как это облегчает поиск и замену оборудования и комплектующих, а также подготовку обслуживающего персонала.
Ключевыми игроками на рынке автоматизации являются ABB, General Electric, Honeywell Solutions, Emerson Electric, Siemens AG, Schneider Electric, Mitsubishi Electric Corporation,  Rockwell Automation и Yokogawa Electric.

Предложений о разработки алгоритмов с помощью методов обучения с подкреплением в свободных источниках лидирующих компаний в автоматизации нет. Но, предполагается, что лидеры области предлагают такие услуги для предприятий высокого ценового сегмента, и стоимость данной услуги -- велика. Помимо этого ключевые игроки готовы работать с оборудованием автоматизации только собственной поставки, что не всегда удобно для компаний низкого и среднего ценового сегмента.


По охватываемым областям промышленности рынок систем автоматизации подразделяется на следующие сегменты: аэрокосмический и оборонный, автомобильный, химический, энергетический, продуктовый, здравоохранение, металлургия, нефтегазовый.

В промышленном производстве обучение с подкреплением предлагается использовать в процессах, где требуются сложные навыки принятия решений и регулирования, особенно, когда необходимо справляться с изменениями в динамической среде. Например, в процессе эксплуатации параметры оборудования могут изменятся. Другой пример применения алгоритмов обучения с подкреплением -- это разработка алгоритмов управления, используя цифровые двойники. Исследователями из лаборатории промышленного искусственного интеллекта Hitachi America разработали виртуальный цех как двумерную матрицу и использовали алгоритмы обучения с подкреплением для многократного взаимодействия с этой виртуальной средой. По результатам моделирования, исследователи смогли определить лучшую настройку для повышения производительности работы цеха и сокращения задержек поставок.


%Потребность в операционной эффективности, быстрорастущий интерес к Интернету вещей (IoT) и облачной автоматизации, растущий спрос на умные фабрики, групповая настройка агрегатов, синхронизация цепочки поставок, рост НИОКР, инновации в области искусственного интеллекта и развитие коммуникационных технологий M2M являются одними из ключевых факторов интереса к нетрадиционных алгоритмам управления систем. 


\textit{Анализ конкурентов.} Прямых аналогов в выделенном сегменте рынка нет. Потенциальными конкурентами выступают научные лабораторий автоматизации при университетах. Например, лаборатория систем автоматизированного проектирования (САПР) в Санкт-Петербургском Политехническом Университете Петра Великого. Основным преимуществом которой, является наличие учебных стендов, которые могут понадобиться на этапе тестирования алгоритмов. Но данная лаборатория не специализируется в современных методах машинного обучения, поэтому предполагается, что угроза появления конкурента низкая. Проведен конкурентный анализ по модели Партера. Результаты представлены в таблице \ref{tab:part}.
\begin{table}[h!]
	\centering
	\caption{Конкурентный анализ по модели Партера}
	\begin{tabular}{|p{8cm}|p{8cm}|}
		\hline
		Критерий & Вывод \\
		\hline
		Оценка конкурентоспособность товара компании и уровня конкуренции на рынке &  средний уровень угрозы со стороны товаров-заменителей \\
		\hline
		Оценка уровня внутриотраслевой конкуренции & Низкий уровень внутриотраслевой конкуренции \\
		\hline 
		Оценка угрозы входа новых игроков & Высокий уровень угрозы входа новых игроков \\
		\hline
		Оценка  угрозы ухода потребителей  (рыночная власть покупателя) & Низкий уровень угрозы ухода клиентов \\
		\hline
		Оценка угрозы для Вашего бизнеса со стороны поставщиков & Низкий уровень влияния поставщиков \\
		\hline
		
	\end{tabular}
	\label{tab:part}
\end{table}

Будут применены следующие стратегии повышении конкурентноспособности:
дифференциация продукта -- уникальность продукта и его высокое качество и особый подход, своевременное реагирование на потребности рынка -- опережение конкурентов во времени за счет универсальности системы реализации алгоритмов управления. 

\section{План маркетинга}
План маркетинга включает в себя план продаж, товарную политику, ценовую политику и сбытовую политику и рекламные мероприятия.

\textit{План продаж.} С учетом роста рынка, на основе метода экспертных оценок сформирован прогнозный план продаж табл.~\ref{table:plan_proda}. Ожидаемый объем продаж и цена услуги установлена исходя из высокого спроса и оценки эксперта в области автоматизации.
\begin{table}[h!]
	\centering
	\caption{План продаж}
	\begin{tabular}{|c|c|c|c|c|c|}
		\hline
		\multirow{2}*{Показатели}&\multicolumn{4}{|c|}{Квартал}   & \multirow{2}*{Всего}\\
		\cline{2-5}
		& I & II & III & IV & \\ 
		\hline
		\multicolumn{6}{|c|}{\textit{Разработка алгоритма управления}} \\
		\hline
		Ожидаемы объем продаж, ед.  & 2  & 3  & 3  & 3  & 10  \\
		\hline
		Цена с НДС, т.р.  & 700  & 500  & 500 & 500  & - \\
		\hline
		Выручка с НДС, т.р.   & 1 400  & 1 500  & 1 500  & 1 500  & 5 900 \\
		\hline
		Нетто-выручка (без НДС), т.р.  & 1 120  & 1 200  & 1 200  & 1 200  & 4 720 \\
		\hline
		Сумма НДС, т.р   & 280 & 300 & 300  & 300  & 1 180 \\
		\hline
	\end{tabular}
	\label{table:plan_proda}
\end{table}


\textit{Товарная политика.} Предлагаемый комплекс продуктов: программный код на высокоуровневом языке, техническая документация, алгоритм конвертации модели на язык программирования промышленных контроллеров (стандарта МЭК (IEC 61131-3)), а так же система визуализации качества регулирования. Так же удовлетворены условия качества продукта -- продукт полностью соответствует современным стандартам программного обеспечения, имеет сертифицированный уровень защиты информации.
Дизайн и товарный знак продукта будут уточнены в процессе разработки.
Предполагаемое техническое обслуживание включено в стоимость -- обращение в техническую поддержку за консультациями по работе системы, помощь с первоначальной установкой на программируемый логический контроллер (ПЛК), демонстрация режимов.
Гарантийное обслуживание: компания не несет ответственности за проблемы в работе системы, вызванные аппаратными сбоями, но помощь по восстановлению работы системы после аппаратных сбоев включена в стоимость.

\textit{Ценовая политика.} Метод ценообразования: постоянная базовая составляющая. Цена продукции получена с учетом экспертной оценки: 500 000 р + 100 000 р, в случае дополнительных ограничений. В первый квартал добавленная надбавка по рекомендации специалиста, с целью обеспечить окупаемость продукта. Скидки не предусмотрены. Условия платежа: единоразовый платеж либо рассрочка на 2 месяца. Формы оплаты: банковский перевод, онлайн-платеж. Сроки и условия предоставления кредита: оплата в кредит не предусмотрена.

\textit{Сбытовая политика и рекламные мероприятия.} С учетом специфики компании возможен только один канал сбыта -- прямые продажи. Исходя из данной сбытовой политики, спланированы расходы на рекламную кампанию.
Одна из главных целей рекламных мероприятий - это завоевание доли рынка и повышение значимости оказываемых услуг.

С целью привлечения новых заказчиков и повышения имиджа компании рекламная деятельность будет проводиться в специализированных печатных изданиях (не реже 2 раз в год), в сети Интернет (путем разработки и продвижения собственного сайта), налаживание коммуникативных связей на специализированных конференциях и выставках. В будущем предполагается добавить рекламную деятельность в форме участия в разнообразных массовых мероприятиях в качестве спонсора. Данные по расходам на рекламные мероприятия приведены в табл. \ref{table:plan_adv}.
\begin{table}[h!]
	\centering
	\caption{Расходы на маркетинговые и рекламные мероприятия}
	\begin{tabular}{|l|c|}
		\hline
		Статья & Сумма, рубл. \\
		\hline
		Реклама в журнале <<Control Engineering Россия>>  & 45 000 \\
		\hline
		Реклама в журнале <<Автоматизация в промышленности>> & 20 000 \\
		\hline
		Создание и продвижение сайта компании  & 20 000 \\
		\hline
		Печать буклетов и брошюр & 5 600 \\
		\hline
		Всего & 90 600\\
		\hline
	\end{tabular}
	\label{table:plan_adv}
\end{table}

Предполагаемая величина суммарных расходов на рекламу в первый год составит 90 600 рублей, в последующие годы прогнозируется увеличение расходов на рекламную компанию на 8\% в год с целью стимулирования спроса на услуги.
Так же в последующие годы планируется проведение исследования на предмет целесообразности открытия филиалов в других населенных пунктах, проведение маркетинговых исследований с целью получения анализа о востребованности услуги, мониторинг клиентской базы и усиленный комплекс коммуникативных мероприятий.


\section{План производства}

Ввиду того, что компания занимается оказанием услуг по разработке и внедрению программного обеспечения, то принципиальной необходимости в удобном местоположении офисов и лабораторий отсутствует.  Проект предусматривает аренду одного офисного помещения с мебелью. Средняя цена аренды офисного помещения с мебелью в 40~ м.$^{2}$ в Санкт-Петербурге стоит около 30~ 000 рублей в месяц. 360~ 000~ рублей в год (288~ 000 + 72~ 000~ руб. НДС).

В таблице \ref{tab:plan_mat} представлены расходы на материалы для осуществления проекта по предоставлению данной услуги за первый год.

\begin{table}[h!]
	\small
	\caption{Потребность в расходных материалах за первый год}
	\label{tab:plan_mat}
	\centering
	\begin{tabular}{|c|p{3.8cm}|c|p{2.5cm}|p{2.5cm}|p{2.5cm}|}
		\hline
		№ & Наименование & Кол-во & Сумма с НДС, руб. & Сумма без НДС, руб & Сумма НДС, руб. \\
		\hline
		1 & Бумага офисная, 500 л., А4  & 4 & 1 000 & 800 & 200\\
		\hline
		2 & Ручка шариковая, синяя, упаковка 8 шт. & 3 & 560 & 448 & 112\\
		\hline
		3 & Папка - регистратор, А4 & 4 & 760 & 608 &152\\
		\hline
		\multicolumn{3}{|c|}{Итого в год} & 2 320 & 1 856 & 464 \\
		\hline
		\multicolumn{3}{|c|}{Итого в месяц} & 193 & 155 & 39 \\
		\hline
	\end{tabular}
\end{table}


На коммунальные услуги (электроэнергия, отопление, водопровод, мобильная связь и пр.) планируется потратить 250 000~ руб. за первый год реализации услуг (200~ 000 + 50~ 000~ руб. НДС). В дни, когда необходимо осуществлять выезд на объект к заказчику возникает необходимость посуточной аренды легкового транспортного средства. На аренду автомобиля планируется потратить 100~ 000~ руб. за первый год реализации услуг (80~ 000 + 20~ 000~ руб. НДС).

В таблице \ref{tab:acc} приведены планируемые затраты на приобретения оборудования. При этом доставка оборудования предоставляется магазинам бесплатно.
\begin{table}[h!]
	\caption{Затраты на приобретение оборудования}
	\small
	\centering
	\begin{tabular}{|p{0.4cm}|p{7cm}|c|c|c|}
		\hline
		№ & Наименование & Кол-во & Цена руб./шт. & Итого, руб.  \\
		\hline
		1 & Монитор Asus VS247NR, 1920x1080, LED, черный & 2 & 9 490 & 18 980 \\
		\hline
		2 & MicroXperts [C300-05] W7PRO персональный компьютер, Intel Core i5-4460, RAM 8Gb, HDD 1Tb, DVD±RW & 2 & 44 560 & 89 120 \\
		\hline
		3 & SVEN Standard 310 Combo, клавиатура USB + оптическая мышь USB, черный & 2 & 990 & 1 980 \\
		\hline
		4 & ИБП CyberPower UT650E & 1 & 3 000 & 3 000 \\
		\hline
		5 & Лазерное МФУ Xerox WorkCentre, A4, Сетевое, USB 2.0, принтер/копир/сканер & 1 & 12 000 & 12 000 \\
		\hline
		6 & Коммутатор Cisco SB SG100D-08-EU & 1 & 3 900 & 3 900 \\
		\hline
		\multicolumn{4}{|c|}{Всего} & 128 980 \\
		\hline
	\end{tabular}
	\label{tab:acc}
\end{table}

Согласно учетной политики организации, амортизация рассчитывается линейным способом, по основному оборудованию, по формуле:
\begin{equation*}
	A = S \cdot K,
\end{equation*}
где $А$ – размер месячных амортизационных отчислений;
$S$ – первичная стоимость имущества;
$K$ – норма амортизации.
Норма амортизации рассчитывается по формуле:
\begin{equation*}
	K = \frac{1}{T} \cdot 100\%,
\end{equation*}
где $T$ -- срок полезного использования, указанный производителем оборудования.
В таблице ~\ref{tab:amort} приведены значения годовых амортизационных отчислений.

\begin{table}[h!]
	\caption{Амортизационные отчисления (годовые)}
	\small
	\centering
	\begin{tabular}{|p{0.4cm}|p{6cm}|p{2cm}|p{3cm}|p{3cm}|}
		\hline
		№ & Наименование & Срок полезного использования & Норма амортизации & Сумма амортизационных отчислений, руб.  \\
		\hline
		1 & Монитор Asus VS247NR & 10 & 10 & 1 898 \\
		\hline
		2 &  Персональный компьютер MicroXperts W7PRO Intel Core i5-4460 & 10 & 10 & 8 912 \\
		\hline
		3 & Клавиатура USB + оптическая мышь USB & 5 & 20 & 396 \\
		\hline
		4 & ИБП CyberPower UT650E & 10 & 10 & 300 \\
		\hline
		5 & Лазерное МФУ Xerox WorkCentre & 10 & 10 & 1 200 \\
		\hline
		6 & Коммутатор Cisco SB SG100D-08-EU & 10 & 10 & 390 \\
		\hline
		\multicolumn{4}{|c|}{Всего} & 13 096 \\
		\hline
	\end{tabular}
	\label{tab:amort}
\end{table}

Годовая амортизация с основных средств составит 13~ 100~ рублей, тогда ежемесячная амортизация составит 1090 рублей.

\textit{Инвестиционные затраты}. Произведена оценка общих инвестиционных затрат, равная суммарной потребности в инвестициях на создание предприятия (инвестиционные затраты на основные средства и предпроизводственные расходы) и потребности в инвестициях для текущей деятельности (оборотные активы, необходимые для формирования начальных товарно-материальных запасов и др.). Результаты представлены в \taref{tab:my_sv}.

\begin{table}[h!]
	\caption{Перечень необходимого оборудования}
	\small
	\label{tab:my_sv}
	\centering
	\begin{tabular}{|c|p{2.6cm}|p{2.1cm}|p{2.5cm}|p{2.5cm}|p{1.8cm}|p{2cm}|}
		\hline
		№ & \small{Наименование} & \small{Способ получения} & \small{Стоимость без НДС, руб.} & \small{Стоимость вкл. НДС, руб.}& \small{Сумма НДС, руб.} & \small{Период получения}\\
		\hline
		1 & \small{Основное оборудование} & Покупка & 103 184 & 128 980 & 25 796 & май 2021 \\
		\hline
		2 & \small{Транспортное средство} & Аренда & 80 000 & 100 000 & 20 000 & май 2021\\
		\hline
		\multicolumn{3}{|c|}{Итого в год} & 183 184 & 228 980 & 36 637 & \\
		\hline
		\multicolumn{3}{|c|}{Итого в месяц} & 15 265 & 19 082 & 3 053 & \\
		\hline
	\end{tabular}
	
\end{table}

Процесс предоставления услуги может быть описан в 5 шагов, представленных в таблице \ref{tab:step1}.

\begin{table}[h!]
	\caption{Характеристика производственных операций}
	\label{tab:step1}
	\small
	\centering
	\begin{tabular}{|p{0.4cm}|p{4.2cm}|p{4.2cm}|p{2.5cm}|p{2.5cm}|}
		\hline
		№ & Наименование выполняемых операций & Наименование используемого оборудования & Объем продукции на выходе & Кол-во занятых чел.\\
		\hline
		1 & Анализ процессов оборудования. Анализ целесообразности применения методов обучения с подкреплением & Персональный компьютер, аренда автомобиля & 1 & 2\\
		\hline
		2 & Реализация алгоритма управления в режиме <<обучение>> & Персональные компьютер & 1 & 1\\
		\hline
		3 & Запуск алгоритма управления в режиме <<обучение>> на оборудовании заказчика & Персональные компьютер, аренда автомобиля & 1 & 1\\
		\hline
		4 & Тестовая конвертация алгоритма к языку МЭК & Персональные компьютер & 1 & 1\\
		\hline
		5 & Подготовка документации & Персональные компьютер, принтер & 1 & 1\\
		\hline
	\end{tabular}
\end{table}


Для реализации проекта необходима линейная организационная структура, которая идеально отвечает вызовам рынка, так как оперативно реагирует на изменения. Для выполнения всех трудовых функций на перспективу ближайших пяти лет достаточно трех человек: главный инженер, ведущий Data Science-специалист, младший инженер, при условии, что в обязанности главного инженера включены управленческие функции. В таблице \ref{tab:salary} представлены затраты на оплату труда рабочих, считая управленческие затраты включенными в основную заработную плату главного инженера. 

\begin{table}[h!]
	\caption{Затраты на оплату труда персонала}
	\small
	\label{tab:salary}
	\centering
	\begin{tabular}{|p{3cm}|p{2cm}|p{3cm}|p{3cm}|p{3cm}|}
		\hline
		Должность & Кол-во человек & Заработная плата & Отчисления на социальные нужды & Итог (з/п + отчисления), руб \\
		\hline
		Главный инженер & 1 & 80 000 & 24 000 & 104 000 \\
		\hline
		Ведущий Data Science-специалист & 1 & 70 000 & 21 000 & 91 000 \\
		\hline
		Младший инженер & 1 & 10 000 & 3 000 & 13 000 \\
		\hline
		\multicolumn{2}{|c|}{Итого в месяц} & 160 000 & 48 000 & 208 000 \\
		\hline
		\multicolumn{2}{|c|}{Итого в год} & 1 920 000 & 576 000 & 2 496 000\\
		\hline
	\end{tabular}
\end{table}

В результате расчетов, общий фонд оплаты труда (ФОТ) за год составит 1~ 920~ 000~ рублей, социальные отчисления 576~ 000~ рублей. 

\section{Финансовый план}

Определим себестоимость усредненной услуги как отношение суммы материальных затрат, без учета НДС, к объему производства за год:

\begin{multline*}
	\text{Себестоимость услуги} = \frac{\Sigma_{\text{расходы}}}{V_{\text{производства}}}= \\
	= \frac{1 856+1 920 000 + 200 000 + 350 000}{10} = 247 186 \text{ рублей}
\end{multline*}

Развитие проекта предполагается за счет заемных средств путем оформления кредита.
Наиболее оптимальным вариантом для работы компании станет оформление кредита, например, в ПАО Банк «ФК Открытие». Сумма кредита составляет 500 000 руб. исходя из первоначальных затрат в проект (единовременные затраты на оборудование и необходимый запас денежных средств на первый период для осуществления текущей деятельности).
Ставка по кредиту составит 5.5\%, срок кредита – 24~ месяцев, ежемесячный платеж равен 22~ 048~ рублей. При данных условиях привлечения денежных средств переплата по кредиту составит 29~ 152~ рубля, выплаты за весь срок кредита составит 529~ 152~ руб.

С учетом приведенных расходов в разделе план производства и динамики роста объема продаж, при условии сохранения стоимости и себестоимости услуг, построен прогноз доходов и расходов на 4 квартала с целью определения финансового результата проекта, таблица \ref{table:plan_prib_}.
В данной таблице приведены доходы и расходы организации без НДС во избежание искажения показателей управленческого учета.



\begin{table}[h!]
	\centering
	\small
	\caption{План прибылей и убытков на 4 квартала}
	\begin{tabular}{|l|c|c|c|c|c|}
		\hline
		\multirow{2}*{Показатели, тыс. руб }&\multicolumn{4}{|c|}{Квартал}   & \multirow{2}*{Всего}\\
		\cline{2-5}
		& I & II & III & IV & \\ 
		
		\hline
		1. Выручка от реализации & 1 120  & 1 200  & 1 200  & 1 200  & 4 720 \\
		\hline
		2. Себестоимость & 494.37 & 741.56 & 741.56 & 741.56 & 2472 \\
		\hline
		3. Затраты & 891.74 & 891.74 & 891.74 & 891.74 & 3 567 \\
		\hline
		3.1. Затраты на материалы & 0.46 & 0.46 & 0.46 & 0.46 & 1.86\\
		\hline
		3.2. Амортизация & 3.27 & 3.27 & 3.27 & 3.27 & 13.1\\
		\hline
		3.3. Затраты на оплату труда с отч. & 624 & 624 & 624 & 624 & 2 496\\
		\hline
		3.4. Общепроизводственные затраты & 122 & 122 & 122 & 122 & 488\\
		\hline
		3.4.1. Аренда помещения & 72 & 72 & 72 & 72 & 288 \\
		\hline
		3.4.2. Коммунальные услуги & 50 & 50 & 50 & 50 & 200\\
		\hline
		3.5. Транспортные расходы & 20 & 20 & 20 & 20 & 80\\
		\hline
		4. Валовая прибыль (1-3) & 228.26 & 308.26 & 308.26 & 308.26 & 1 153\\
		\hline
		5. Коммерческие затраты & 22.65 & 22.65 & 22.65 & 22.65 & 91\\
		\hline
		6. Прибыль от продаж (4-5) & 205.61 & 285.61 & 285.61 & 285.61 &1 062 \\
		\hline
		7. Выплаты по кредиту & 66.14 & 66.14 & 66.14 & 66.14 & 264 \\
		\hline
		8. Прибыль до налогообл. (6-7) & 139.47 & 219.47 & 219.47 & 219.47 & 798 \\
		\hline
		9. Налог на прибыль, 20\% & 27.89 & 43.89 & 43.89 & 43.89 & 159 \\
		\hline
		9. Чистая (нераспр.) прибыль & 111.58 & 175.58 & 175.58 & 175.58 & 638\\
		\hline
	\end{tabular}
	\label{table:plan_prib_}
\end{table}

В работе рассчитана чистая текущая стоимость проекта (NPV – Net Present Value), как разность дисконтированных денежных потоков поступлений и платежей, производимых в процессе реализации проекта за весь инвестиционный период. Значения приведены в таблице \ref{table:plan_mov}.


\begin{table}[h!]
	\centering
	\small
	\caption{План прибылей и убытков на 4 квартала}
	\label{table:plan_mov}
	\begin{tabular}{|p {9cm}|c|c|c|c|}
		\hline
		\multirow{2}*{Показатели, тыс. руб }&\multicolumn{4}{|c|}{Квартал} \\  
		\cline{2-5}
		& I & II & III & IV \\ 
		
		\hline
		1. Поступление денежных средств & 1 900  & 1 500  & 1 500  & 1 500   \\
		\hline
		1.1. Поступление денежных средств от продажи продукции & 1 400 & 1 500 & 1 500 & 1 500 \\
		\hline
		1.2. Поступление денежных средств от кредита & 500 & 0 & 0 & 0\\
		\hline
		2. Производственные и общехозяйственные расходы & 624.65 & 624.65 & 624.65 & 624.65\\
		\hline
		2.1. Оплата труда & 480 & 480 & 480 & 480\\
		\hline
		2.2. Оплата общепроизводственных расходов & 122 & 122 & 122 & 122\\
		\hline
		2.3. Оплата коммерческих рассходов & 22.65  & 22.65 & 22.65 & 22.65 \\
		\hline
		2.4. Транспортные расходы & 20  & 20 & 20 & 20 \\
		\hline
		3. Покупка оборудования & 103.184 &0 &0 &0 \\
		\hline
		4. Уплата налогов & 171.89 &187.89 &187.89 &187.89 \\
		\hline
		4.1. Отчисления на соц.нужды & 144 &144 & 144 & 144\\
		\hline
		4.2. Налог на прибыль & 27.89 & 43.89 & 43.89 & 43.89 \\
		\hline
		5. Всего отток денежных средств (2+3+4) & 905.72 & 812.54 & 812.54 & 812.54 \\
		\hline
		6. Погашение кредита & 66.14 & 66.14 & 66.14 & 66.14 \\
		\hline
		7. Чистый денежный поток (1-5-6) & 924.14 & 621.32 & 621.32 & 621.32 \\
		\hline
		8. Дисконтированный денежный поток (20\%) & 739.31 & 431.47 & 359.56 & 300.15 \\
		\hline
		9. Дисконтированный денежный поток нарастающим итогом (NVP) & 739.31 & 1 170.78 & 1 530,34 & 1 830.49 \\
		\hline
	\end{tabular}
\end{table}

Вычислен дисконтированный период окупаемости инвестиций (срок возврата):
\begin{equation*}
	T_{\text{ок}} = x + \frac{NVP_x} {\text{ЧДД}_{x-1}} = 2 + \frac{339.31}{593.23} = 2.57 (\text{месяца})
\end{equation*} 
где $x$ -- последний месяц/год, когда NVP < 0; $NVP_x$ -- значение NVP в этом месяце/году;  ЧДД$_{x-1}$ -- значение ЧДД в следующем периоде; $T_{ok}$ -- срок окупаемости. 
Очевидно, что чем меньше период возврата инвестиций, тем более экономически привлекательным является проект. В нашем случае NVP в первом же квартале имеет положительное значение, поэтому потребовалось дополнительно рассчитать значение NVP для первых трех месяцев предоставления услуги. Первые 2 месяца NVP имеет отрицательное значения. 

Чистый дисконтированный поток за четыре квартала с учетом дисконтирования под 20\% составит 1 830 тыс. руб. Положительное значение NPV свидетельствует о целесообразности принятия решения о финансировании и реализации проекта. При сравнении нескольких инвестиционных вариантов показателя внутренней рентабельности проекта (IRR) служит критерием отбора более эффективного варианта. На данном этапе нет необходимости рассчитывать данный параметр. Так же стоит отметить, что по полученным расчетам окупаемость проекта наступает в 1 квартале, что в действительности очень тяжело осуществимо, но так как расчет производился в учебных целях, работа подтверждает экономическую целесообразность оказания услуги по разработке алгоритмов управления с использованием методом обучения с подкреплением. 

\newpage
Вывод по главе \thechapter. В ходе составления бизнес-плана по коммерциализации проекта был сделан вывод о целесобразности оказания услуги разработки регуляторов на базе методов обучения с подкреплением для малого и среднего бизнеса.

\newpage
%% Вспомогательные команды - Additional commands
%
%\newpage % принудительное начало с новой страницы, использовать только в конце раздела
%\clearpage % осуществляется пакетом <<placeins>> в пределах секций
%\newpage\leavevmode\thispagestyle{empty}\newpage % 100 % начало новой страницы