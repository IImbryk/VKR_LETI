\chapter*{Заключение} \label{ch-conclusion}
\addcontentsline{toc}{chapter}{\hspace{33pt}Заключение}	% в оглавление 

В работе произведен анализ современного подхода из области искусственного интеллекта -- обучения с подкреплением, с точки зрения теории автоматического управления. 
В ходе работы установлено, что специалистами по автоматизации методы обучения с подкреплением могут быть применены для решения задач оптимального и адаптивного управления. Главным преимуществом методов обучения с подкреплением является то, что с их помощью возможно осуществить синтез регулятора не имея предварительной информации о модели объекта управления, а ограничившись симулятором типа <<черный ящик>> или предоставленной возможностью взаимодействия с объектом.

В работе реализована серия экспериментов по разработке регуляторов с использованием методов обучения с подкреплением. Разработаны регуляторы для управления обратным маятником, производством пенициллина и управлением ростом опухоли. В последнем случае качество регулирование не уступает оптимальному регулятору и регулятору с прогнозирующей моделью. Проведенные эксперименты только подтверждают важность и применимость методов обучения с подкреплением для сложных систем, в частности для задач управления биологическими процессами. 

В дальнейшей работе необходимо рассмотреть вопросы устойчивости регуляторов на базе обучения с подкреплением.