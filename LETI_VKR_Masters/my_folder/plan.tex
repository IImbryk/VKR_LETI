\newpage
\begin{center}
\textbf{КАЛЕНДАРНЫЙ ПЛАН ВЫПОЛНЕНИЯ \\ ВЫПУСКНОЙ КВАЛИФИКАЦИОННОЙ РАБОТЫ}\\
\end{center}
\begin{flushright}
Утверждаю\\
Заф. кафедры АПУ\\
\underline{\hspace{2.5cm}} Шестопалов М. Ю.\\
«\underline{\hspace{0.7cm}}»\underline{\hspace{4cm}}2021 г.
\vspace{1cm}
\end{flushright}

\begin{flushleft}
Студентка Шпаковская И.И. \hspace{7cm} Группа 5391\\
\vspace{0.3cm}
Тема работы: Применение методов обучения с подкреплением в задачах управления.\\
\vspace{0.2cm}
\end{flushleft}


{\centering
\begin{tabular}{|>{\raggedright\arraybackslash}m{8mm}|m{125mm}|m{25mm}|}
	\hline
	\multicolumn{1}{|>{\centering\arraybackslash}m{8mm}|}{№ п/п} 
	& \multicolumn{1}{>{\centering\arraybackslash}m{125mm}|}{Наименование работ} 
	& \multicolumn{1}{>{\centering\arraybackslash}m{25mm}|}{\mbox{Срок} \mbox{выполнения}}\\
	\hline
	\centering {1} & Обзор литературы по теме работы & 08.02 -- 20.02 \\
	\hline
	\centering {2} & Изучение общей терминологии и методов области обучения с подкреплением  & 20.02 -- 09.03 \\
	\hline
	\centering {3} & Изучение программных продуктов, предоставляемых MATLAB  & 09.03 -- 25.03 \\
	\hline
	\centering {4} & Анализ схемы алгоритмов обучения с подкреплением, сравнение с базовыми адаптивными схемами  & 25.03 -- 26.04 \\
	\hline
	\centering {5} & Реализация регуляторов на основе обучения с подкреплением для математического маятника, модели распространения опухоли и производства пеницилина  & 26.04 -- 10.05 \\
	\hline
	\centering {6} & Разработка бизнес-плана по коммерциализации проекта  & 10.05 -- 12.05 \\
	\hline
	\centering {7} & Оформление пояснительной записки  & 12.05 -- 25.05 \\
		\hline
\end{tabular}
}

\vspace{0.5cm}

\begin{flushleft}
	Студентка \hspace{6cm} \underline{\hspace{3cm}} \hspace{1cm}  Шпаковская И.И. \\ 
	\vspace{5mm}
	Руководитель \hspace{0,5cm} д.т.н., профессор \hspace{1cm} \underline{\hspace{3cm}}\hspace{1cm}  Душин С.Е.\\ 
\end{flushleft}

\thispagestyle{empty} % выключаем отображение номера для этой страницы