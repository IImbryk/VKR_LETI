\chapter*{Введение} % * не проставляет номер
\addcontentsline{toc}{chapter}{\hspace{33pt} Введение} % вносим в содержание
Классические и современные методы теории автоматического управления направленны на разработку систем управления на основе математического и компьютерного моделирования объекта, путем описания физики процессов, как правило, нелинейных динамических объектов. В то же время, с учетом скорости развития вычислительных ресурсов, возникает вопрос о возможности применения накопленных об объекте данных для разработки и уточнения систем регулирования.
Такие области как адаптивное, нейросетевое, нечеткое управление и идентификация позволяют не ограничиваться представлением объекта в форме пространства состояний или дифференциальных уравнениях, а учитывают тот факт, что объект представлен набором данных. 

Разработка адаптивных регуляторов характерна тем, что процесс подстройки параметров регулятора возможен в режиме онлайн, то есть, используя данные, измеренные в реальном времени. Адаптивные регуляторы могут удовлетворять некоторым условиям оптимальности, но в то же время, адаптивные контроллеры, как правило, не проектируются как оптимальные в смысле минимизации заданного функционала качества. Так же адаптивные регуляторы требуют исследования структуры объекта управления. Косвенные адаптивные регуляторы используют методы идентификации, чтобы сначала определить параметры системы, а затем использовать полученную модель для нахождения оптимального управления. 
Оптимальное управление обычно проектируются в автономном режиме (режиме off-line) путем решения уравнений Гамильтона–Якоби–Беллмана, решение которых часто затруднительно.

Возникает вопрос о возможности применения современных методов, которые способны решать задачи управления с учетом оптимизации функционала, адаптируясь к изменением параметров объекта управления.

В большинстве современных исследований по обучению машин, распознаванию изображений и искусственным нейронным сетям рассматривается подход обучения с учителем, цель которого -- обобщать, располагая лишь фиксированным набором данных, с ограниченным количеством примеров. Тогда как существует подход, называемый \textit{обучение с подкреплением}, где в основе лежит идея об активном взаимодействии со средой методом проб и ошибок, с целью определения последовательности действий, максимизирующую некоторую награду. Другими словами, алгоритмы обучения с подкреплением построены на идее, что эффективные управляющие значения должны запоминаться с помощью сигнала подкрепления для повторного использования.

Большая часть теории, лежащей в основе обучении с подкреплением, основана на предположении гипотезы вознаграждения, которая вкратце утверждает, что все цели и задачи агента могут быть объяснены одним скаляром, называемым вознаграждением. Более формально гипотеза вознаграждения представлена далее:
все, что мы подразумеваем под целями и задачами, можно хорошо представить как максимизацию ожидаемого значения совокупной суммы полученного скалярного сигнала (называемого вознаграждением).


\textbf{Актуальность исследования} заключается в интересе со стороны сообщества инженеров по автоматизации в использовании новых подходов на основе алгоритмов машинного обучения.


\textbf{Объект исследования} --- методы обучения с подкрепленим.

\textbf{Предмет исследования} --- определить связь методами обучения с подкреплением с адаптивными и оптимальными подходами в теории автоматического управления.

\textbf{Цель исследования} -- проанализировать современные методы искусственного интеллекта, а именно метод обучения с подкреплением, на предмет применимости для решения задач автоматизированного управления сложных систем.
  
В соответствии с целью исследования, определяются следующие \textbf{задачи работы}:

\begin{itemize}
	\item Проанализировать терминологию и математический аппарат области обучения с подкреплением, свойственный сообществу искусственного интеллекта.
	\item Формализовать терминологию, методов и алгоритмов области обучения с подкреплением в рамках теории автоматического управления.
	\item Привести сравнительный анализ концепции разработки регуляторов на основе обучения с подкрепленим с методами адаптивного, оптимального управления.
	\item Провести ряд программных экспериментов для сложных объектов управления. Качественно установить применимость регулятора на основе обучения с подкреплением.
\end{itemize} 

Выдвигается следующая \textbf{гипотеза} -- обучения с подкреплением напрямую связанно с теорией оптимального управления, поэтому терминология и методы обучения с подкреплением напрямую связанны с оптимальным управлением. Общая структурная схема регулирования на базе обучения с подкреплением соотвествует известным в теории адаптивного управления подходам. Методы обучения с подкреплением позволяют получать качество регулирования сложными техническими системами на том же уровне, что и численные методы оптимального управления. 


\textbf{Теоретическая и методологическая база исследования}. Теоретической основой выпускной квалификационной работы послужили исследования Дмитрия Бертсекаса в области динамического программирования и оптимального управления, а так же работы Ричарда Саттона и Эндрю Барто в области обучения с подкреплением.
Практическая часть работы выполнялась на основании научных статей по разработке математических моделей в дифференциальных уравнениях, а так же открытых интернет источников по реализации алгоритмов обучения с подкреплением на языке Python и MATLAB.
При подготовке ВКР были использованы материалы таких учебных дисциплин, как <<Оптимальные системы управления>>, <<Современные методы теории управления>>, <<Адаптивные системы управления>>, <<Нейросетевые системы управления>>, «Коммерциализация результатов научных исследований и разработок».



%% Вспомогательные команды - Additional commands
%\newpage % принудительное начало с новой страницы, использовать только в конце раздела
%\clearpage % осуществляется пакетом <<placeins>> в пределах секций
%\newpage\leavevmode\thispagestyle{empty}\newpage % 100 % начало новой строки