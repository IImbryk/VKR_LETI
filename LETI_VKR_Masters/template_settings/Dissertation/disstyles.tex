%%% Изображения %%%
\graphicspath{{images/}{Dissertation/images/}}         % Пути к изображениям

%%% Макет страницы %%%
% Выставляем значения полей (ГОСТ 7.0.11-2011, 5.3.7)
\makeatletter
\geometry{a4paper,top=2cm,bottom=2cm,left=3cm,right=1.5cm,
	headsep=0.5cm, %отступ от колонтитула от живописного поля
	head=1cm, %верхняя граница колонтитула
	headheight=1cm,
	%nofoot,
%includefoot,
	nomarginpar
%	,showframe
} 
%\setlength{\topskip}{0pt}   %размер дополнительного верхнего поля
\makeatother

%%% Интервалы %%%
%% По ГОСТ Р 7.0.11-2011, пункту 5.3.6 требуется полуторный интервал
%% Реализация средствами класса (на основе setspace) ближе к типографской классике.
%% И правит сразу и в таблицах (если со звёздочкой) 
%\DoubleSpacing*     % Двойной интервал
\OnehalfSpacing*    % Полуторный интервал % * to force it in the floats
%\setSpacing{1.42}   % Полуторный интервал, подобный Ворду (возможно, стоит включать вместе с предыдущей строкой)

%https://tex.stackexchange.com/questions/65849/confusion-onehalfspacing-vs-spacing-vs-word-vs-the-world/276516#276516
%https://tex.stackexchange.com/questions/13742/what-does-double-spacing-mean
%https://tex.stackexchange.com/questions/30073/why-is-the-linespread-factor-as-it-is/30114#30114

%A possible definition of \onehalfspacing and \doublespacing is that the ratio between font size and \baselineskip should be 1.5 resp. 2.....
%\baselineskip (vertical skip between the base lines of two successive lines of type) of XXpt. 


%%% Выравнивание и переносы %%%
%% http://tex.stackexchange.com/questions/241343/what-is-the-meaning-of-fussy-sloppy-emergencystretch-tolerance-hbadness
%% http://www.latex-community.org/forum/viewtopic.php?p=70342#p70342
\tolerance 1414
\hbadness 1414
\emergencystretch 1.5em % В случае проблем регулировать в первую очередь
\hfuzz 0.3pt
\vfuzz \hfuzz
%\raggedbottom
%\sloppy                 % Избавляемся от переполнений
\clubpenalty=10000      % Запрещаем разрыв страницы после первой строки абзаца
\widowpenalty=10000     % Запрещаем разрыв страницы после последней строки абзаца

%%% Блок управления параметрами для выравнивания заголовков в тексте %%%
\newlength{\otstuplen}
\setlength{\otstuplen}{\theotstup\parindent}
\ifnumequal{\value{headingalign}}{0}{% выравнивание заголовков в тексте
    \newcommand{\hdngalign}{\centering}%{\centering}                % по центру
    \newcommand{\hdngaligni}{}% по центру
    \setlength{\otstuplen}{0pt}
}{%
    \newcommand{\hdngalign}{}     
    %\setlength{\otstuplen}{1cm}           % по левому краю
    \newcommand{\hdngaligni}{\hspace{\otstuplen}}      % по левому краю
} % В обоих случаях вроде бы без переноса, как и надо (ГОСТ Р 7.0.11-2011, 5.3.5)







%%% Оглавление %%%

%\renewcommand{\cftchapterdotsep}{\cftdotsep}                % отбивка точками до номера страницы начала главы/раздела
\renewcommand{\cftsectiondotsep}{\cftnodots}


%% снятие жирности %%
%\cftKleader defines the leader between the title and the page number; it can be
%changed by \renewcommand. The spacing between any dots in the leader is controlled
%by \cftKdotsep
\makeatletter
\renewcommand{\cftchapterpagefont}{\normalfont}        % нежирные номера страниц у глав в оглавлении
%\renewcommand{\cftchapterleader}{\cftdotfill{\cftchapterdotsep}}% нежирные точки до номеров страниц у глав в оглавлении
\renewcommand{\cftchapterfont}{}                       % нежирные названия глав в оглавлении
\renewcommand{\cftchapterpagefont}{}                       % нежирные названия номеров глав в оглавлении
\makeatother


%% Форматирование SPbPU %%
% Варианты форматирования
%https://tex.stackexchange.com/questions/394227/memoir-toc-indent-the-second-line-by-numberspace-width-in-the-previous-line-or



%% работа с расстояниями между точками, переносами слов
\makeatletter
\renewcommand{\cftdotsep}{0.1}
%\renewcommand{\@dotset}{0.1} %old macro DOES NOT WORK
\setpnumwidth{2.84538em} %2.84538em = 1cm  
%\renewcommand{\@pnumwidth}{0em} %old macro
%%\setrmarg > \setpnumwidth !!!
\setrmarg{2.84539em}
%set large number to restrict hyphenation
%plus1fil makes the distance between the words smaller!
%it helps to make the equal indent
\makeatother

%\usepackage{tocloft}    % tocloft for table of contents style

%% отступы %%
\makeatletter
\renewcommand{\cftchapterbreak}{}        % set a page break before rather than after the entry
%\renewcommand{\cftparskip}{10em} % эта настройка не работает, вместо неё изменен \parskip непостредственно перед \tableofcontents
\setlength{\cftbeforechapterskip}{0pt plus 0pt} %delete skip after chapter block (last section) %%%<-SPbPU pure
\setlength{\cftbeforepartskip}{0pt plus 0pt} %delete skip after chapter block (last section) %%%<-SPbPU pure
\makeatother



%% Продолжение редактирования оглавления настройками CandDoctDiss %%		


\ifnumgreater{\value{headingdelim}}{0}{%
	%<- SPbPU точка после номера страницы
	\renewcommand\cftchapteraftersnum{\enspace\enspace}       % добавляет точку с пробелом после номера раздела в оглавлении
	\renewcommand\cftchapteraftersnumb{\enspace \enspace}
	\renewcommand\cftsectionpresnum{\enspace}
	\renewcommand\cftsectionaftersnumb{\enspace}%\textperiodcentered\space} %\enspace - настоящий пробел, \space не работает
	%\renewcommand\chapternumberlinebox[2]{#2}
}{}
\ifnumgreater{\value{headingdelim}}{1}{%%%<-SPbPU 
		   	\renewcommand\cftsectionpresnum{\enspace}       % добавляет smth перед number %выравнивает в box
	% точка после номера страницы
	\renewcommand\cftsectionaftersnum{\enspace}       % добавляет точку с пробелом после номера подраздела в оглавлении
	 last is \hfil !
	   	\renewcommand\cftsectionaftersnumb{\enspace }       % добавляет точки перед названием %можно удалить пробел
	\renewcommand\cftsubsectionaftersnum{\space}    % добавляет точку с пробелом после номера подподраздела в оглавлении
	\renewcommand\cftsubsubsectionaftersnum{\enspace} % добавляет точку с пробелом после номера подподподраздела в оглавлении
	\AtBeginDocument{% без этого polyglossia сама всё переопределяет
		\setsecnumformat{\csname the#1\endcsname.\space}
		%\setsecnumformat{\csname the#1\endcsname:\quad}
	}
}{%
	\AtBeginDocument{% без этого polyglossia сама всё переопределяет
		\setsecnumformat{\csname the#1\endcsname\quad} %
	}
}

\renewcommand*{\cftappendixname}{\appendixname\space} % Слово Приложение в оглавлении


%%% Различные варианты форматирования SPbPU %%%

%% форматирование отсупов до номеров страниц стр. 151 мемуара !!!
%\renewcommand*{\cftsectionnumwidth}{} %выставление абсолютного значения
%\addtolength{\cftsectionnumwidth}{5em} %не работает

%убираем фиксированные размеры of the box %%%<-SPbPU pure
\AtBeginDocument{%
\renewcommand\numberlinebox[2]{#2} % for sections %%%<-SPbPU pure
\renewcommand\chapternumberlinebox[2]{#2} % for chapters 
%\newcommand\chapternumberlinebox[2]{%
%	\hb@xt@#1{#2\hfil}}
%
%\newcommand\chapternumberlinebox[2]{%
%	#1{\hfil#2}}

%\numberlinebox{hlengthi}{hcodei} %выставление абсолютного значения
%\chapternumberlinebox{hlengthi}{hcodei} %выставление абсолютного значения
}

%убираем растояния до \cftsectionpresnum в размере одного абзацного отступа %%%<-SPbPU pure
%\cftsetindents{hkindi}{hindenti}{hnumwidthi}


%https://tex.stackexchange.com/questions/306851/multiline-unnumbered-chapter-in-table-of-contents
%https://tex.stackexchange.com/questions/40022/customized-table-of-contents-same-indentation-for-every-line-of-multi-line-titl
%\parindent % standart padding
% это здорово экономит место, но нужно тогда синхронизировать стиль обычных отступов в перечислениях
% недостаток - не видно выравнивания по первому слову в названии предыдущего раздела
\AtBeginDocument{
	\cftsetindents{chapter}{0pt}{% первая строка
		-0.05em} % последующие строки от первой
	\cftsetindents{section}{%
		0pt
%3.5ex plus 1ex minus .2ex
	}{%
		\parindent
%2.3ex plus .2ex
}
	\cftsetindents{subsection}{%
	0pt}{% 
		\parindent}
	\cftsetindents{subsubsection}{%
		0pt}{% 
		\parindent}
}



%%% Колонтитулы %%%
% Порядковый номер страницы печатают на середине верхнего поля страницы (ГОСТ Р 7.0.11-2011, 5.3.8)
%сделаем справа внизу
\makeatletter
\makeevenhead{plain}{}{}{\thepage}
%\makeoddhead{plain}{}{}{\thepage}
\makeevenfoot{plain}{}{}{}
%\makeoddfoot{plain}{}{}{}
%\makeoddfoot{plain}{}{}{\thepage} 
\pagestyle{plain}

%%% добавить Стр. над номерами страниц в оглавлении
%%% http://tex.stackexchange.com/a/306950
\newif\ifendTOC
%
\newcommand*{\tocheader}{
%\ifnumequal{\value{pgnum}}{1}{%
%    \ifendTOC\else\hbox to \linewidth%
%      {\noindent{}~\hfill{Pages}}\par%
%      \ifnumless{\value{page}}{3}{}{%
%        \vspace{0.5\onelineskip}
%      }
%      \afterpage{\tocheader}
%    \fi%
%}{}%
}%


%epigraph with DOI
%\usepackage{quotchap}




%%% SPbPU %%% Оформление заголовков глав, разделов, подразделов %%%

\newcommand{\printTheAbstract}{%распечатать the Abstract
	\begingroup
	\par
	\renewcommand{\cleardoublepage}{}
	\renewcommand{\clearpage}{}
	\vspace{\theintvl\curtextsize}
	\chapter*{Abstract}
	\endgroup %after chapter in case of inline using
	\thispagestyle{empty}%
}


\makechapterstyle{SPbPUstyle}{%
	\chapterstyle{default}
	\setlength{\beforechapskip}{0pt}
	\setlength{\midchapskip}{0pt} 
	\setlength{\afterchapskip}{1\curtextsize}
	\renewcommand*{\chapnamefont}{\normalfont\bfseries\MakeTextUppercase} %не используется слово <<Глава>>
	\renewcommand*{\chapnumfont}{\normalfont\bfseries\MakeTextUppercase}
%	\renewcommand*{\chaptitlefont}{\normalfont\bfseries\MakeTextUppercase} %не работает \MakeTextUppercase
	\renewcommand\printchaptertitle{\centering\normalfont\bfseries\MakeTextUppercase}% аналог \chaptitlefont
	\renewcommand*{\chapterheadstart}{}
	\ifnumgreater{\value{headingdelim}}{0}{%
		\renewcommand*{\afterchapternum}{\space}   % добавляет точку с пробелом после номера раздела
	}{%
		\renewcommand*{\afterchapternum}{\space}     % добавляет точку и \space (\quad) после номера раздела
	} % настройки добавление в СОДЕРЖАНИЕ (по умолчанию название раздела переходит само)
	\renewcommand*{\printchapternum}{\hdngaligni\hdngalign\chapnumfont \thechapter}
	\renewcommand*{\printchaptername}{}
	\renewcommand*{\printchapternonum}{\hdngaligni\hdngalign}
	}
\newcommand{\chapterLight}{\normalfont\MakeTextUppercase\normalsize} %не менять последовательность команд!
\renewcommand*{\printtoctitle}[1]{\normalfont\MakeTextUppercase #1} %слово <<Content>> в стилю chaperLight, по факту убираем \bfseries
%\chapterLight не действует в этой команде
\makeatletter


%\makechapterstyle{SPbPUstylechapname}{% для <<будет вписано слово Глава перед каждым номером раздела в оглавлении>>
%	\chapterstyle{SPbPUstyle}
%	\renewcommand*{\printchapternum}{\chapnumfont \thechapter}
%	\renewcommand*{\printchaptername}{\hdngaligni\hdngalign\chapnamefont \@chapapp} %

%}
\makeatother

\chapterstyle{SPbPUstyle}

%% удалить перенос на новую строку перед командой \chapter
\newcommand{\delnewpagebeforech}{
	%\begingroup
	\renewcommand{\cleardoublepage}{}
	\renewcommand{\clearpage}{}
	\vspace{\theintvl\curtextsize}
	%\endgroup %after chapter in case of inline using
}

%% Оформление шрифтов и отсупов подразделов, подподразделов и подподподразделов

\makeatletter
\setsecheadstyle{\normalfont\bfseries\hdngalign}
\setsecindent{\otstuplen} %отступ от левого края живописного поля
\setbeforesecskip{0.2\curtextsize} %базовые настройки с плюс/минус точностью, что позволяет более гибко располагать рисунки и изображения на странице
\setaftersecskip{0.2\curtextsize}


\setsubsecheadstyle{\normalfont\bfseries\itshape\hdngalign}
\setsubsecindent{\otstuplen}
\setbeforesubsecskip{1\curtextsize}
\setaftersubsecskip{1\curtextsize}

\setsubsubsecheadstyle{\normalfont\itshape\hdngalign}
\setsubsubsecindent{\otstuplen}
\setbeforesubsubsecskip{1\curtextsize}
\setaftersubsubsecskip{1\curtextsize}

%ОLD  ГИА
%\setsubsecheadstyle{\normalfont\hdngalign}
%\setsubsecindent{\otstuplen}
%\setbeforesubsecskip{\intvlS\curtextsize}
%\setaftersubsecskip{\intvlS\curtextsize}
%
%\setsubsubsecheadstyle{\normalfont\hdngalign}
%\setsubsubsecindent{\otstuplen}
%\setbeforesubsubsecskip{\intvlS\curtextsize}
%\setaftersubsubsecskip{\intvlS\curtextsize}
\makeatother

%попытки форматирования \part можно продолжить
%сейчас реализован более простой вариант
\renewcommand{\partnamefont}{\LARGE\MakeTextUppercase}
\renewcommand{\partnumfont}{\LARGE\MakeTextUppercase}
\renewcommand*{\parttitlefont}{\LARGE\MakeTextUppercase}

%[section], чтобы заставить все floats быть до расположиться до окончания подраздела
%\FloatBarrier локальное ограничение, чтобы 
% расставить повсеместно по разделам, то всего лишь подключить [section];
% разрешить до \FloatBarrier размещать foats, то добавить окцию  [above].
\usepackage[above]{placeins} 

\sethangfrom{\noindent #1} %все заголовки подразделов центрируются с учетом номера, как block 

%\ifnumequal{\value{chapstyle}}{1}{%
 %   \chapterstyle{SPbPUstylechapname}
%   \renewcommand*{\cftchaptername}{Глава\space} % будет вписано слово Глава перед каждым номером раздела в оглавлении
%}{}% вместо Chapter \chaptername

%%% Интервалы между заголовками
%\setbeforesecskip{\theintvl\curtextsize}% Заголовки отделяют от текста сверху и снизу тремя интервалами (ГОСТ Р 7.0.11-2011, 5.3.5).
%\setaftersecskip{\theintvl\curtextsize}
%\setbeforesubsecskip{\theintvl\curtextsize}
%\setaftersubsecskip{\theintvl\curtextsize}
%\setbeforesubsubsecskip{\theintvl\curtextsize}
%\setaftersubsubsecskip{\theintvl\curtextsize}


%%% Блок дополнительного управления размерами заголовков
\ifnumequal{\value{headingsize}}{1}{% Пропорциональные заголовки и базовый шрифт 14 пт
	\renewcommand{\normalfont}{\large\bfseries}
	\renewcommand*{\chapnamefont}{\Large\bfseries}
	\renewcommand*{\chapnumfont}{\Large\bfseries}
	\renewcommand*{\chaptitlefont}{\centering\Large\bfseries}
}{}




% ОФОРМЛЕНИЕ Приложений Appendix - Вариант 2 - действующий
%https://stackoverflow.com/questions/717316/how-to-make-appendix-appear-in-toc-in-latex
\makeatletter
\newcommand\appendix@chapter[1]{%
	\refstepcounter{chapter}%
	\def\app@ct{ Приложение \@Alph\c@chapter \space #1 }
	\orig@chapter*{\app@ct}%
	\markboth{\MakeUppercase{\app@ct}}{\MakeUppercase{\app@ct}}
	\addcontentsline{toc}{chapter}{\app@ct}%
}
\let\orig@chapter\chapter
\g@addto@macro\appendix{\let\chapter\appendix@chapter}
\makeatother

%\makeatletter
%\newcommand\appendix@chapter[1]{%
%	\renewcommand*{\chapnamefont}{\normalfont\normalsize\bfseries} %не используется слово <<Глава>>
%	\renewcommand*{\chapnumfont}{\normalfont\normalsize\bfseries}
%	\renewcommand\printchaptertitle{\normalfont\normalsize\bfseries}
%	\renewcommand*{\printchapternum}{\chapnumfont \thechapter}
%	\renewcommand*{\printchaptername}{\hdngaligni\hdngalign\chapnamefont% \@chapapp} %
%	\renewcommand*\thechapter{\arabic{chapter}} % работает
%	\settocdepth{chapter} % выводить только названия Приложений
%	\refstepcounter{chapter}%
%	\def\app@ct{\hfill{}\appendixname{} {}\@arabic\c@chapter %
%	\vspace{\intvlS\curtextsize}
%	\newline #1
%	\vspace{\curtextsize}
%}
	%\orig@chapter*{\app@ct}%
%	\addcontentsline{toc}{chapter}{\appendixname{} \@arabic\c@chapter. %#1}%\app@ct % input to TOC-table
%}
%\let\orig@chapter\chapter
%\g@addto@macro\appendix{\newpage\let\chapter\appendix@chapter\renewcomma%nd*{\afterchapternum}{\par\nobreak\vskip \midchapskip}}
%\makeatother


%https://tex.stackexchange.com/questions/250834/dont-break-page-for-new-chapter-unless-chapter-heading-wont-fully-fit-on-curre
\newcommand{\ContinueChapterBegin}{%
\let\clearpage\relax
\renewcommand*{\chapterheadstart}{%
	\FloatBarrier % make floats stop
\par
\ifartopt % если сверху сраницы, то
% ничего не делать
\else % в противном случае
\vspace{\theintvl\curtextsize} % добавить интервал
\fi
}
}%

\newcommand{\ContinueChapterEnd}{%
	\let\clearpage\newpage
\renewcommand*{\chapterheadstart}{% ничего не делаем
\FloatBarrier % make floats stop
}
}%

\newcommand{\NewPage}{% в случае, если на последней странице приложения есть <<висячая>> таблица или рисунок
\newpage\leavevmode\thispagestyle{empty}\newpage %начать новое приложение с новой страницы 
}%




\makeatletter %настройка отображения floates
\setlength{\@fptop}{0pt}%отключить вертикальное центрирование рисунка/таблицы на странице
%\setlength{\@fpsep}{8pt}%отключить вертикальное центрирование рисунка/таблицы на странице
%\setlength{\@fpbot}{0pt plus 1fil}%отключить вертикальное центрирование рисунка на странице
\makeatother



%%% Счётчики %%%

%% DOI
\newcounter{mychapternumber} 
\newcounter{chapterDOI}

%% Упрощённые настройки шаблона диссертации: нумерация формул, таблиц, рисунков
\ifnumequal{\value{contnumeq}}{1}{%
    \counterwithout{equation}{chapter} % Убираем связанность номера формулы с номером главы/раздела
}{}
\ifnumequal{\value{contnumfig}}{1}{%
    \counterwithout{figure}{chapter}   % Убираем связанность номера рисунка с номером главы/раздела
}{}
\ifnumequal{\value{contnumtab}}{1}{%
    \counterwithout{table}{chapter}    % Убираем связанность номера таблицы с номером главы/раздела
}{}


%%%http://www.linux.org.ru/forum/general/6993203#comment-6994589 (используется totcount)
\makeatletter
\def\formbytotal#1#2#3#4#5{%
    \newcount\@c
    \@c\totvalue{#1}\relax
    \newcount\@last
    \newcount\@pnul
    \@last\@c\relax
    \divide\@last 10
    \@pnul\@last\relax
    \divide\@pnul 10
    \multiply\@pnul-10
    \advance\@pnul\@last
    \multiply\@last-10
    \advance\@last\@c
    \total{#1}~#2%
    \ifnum\@pnul=1#5\else%
    \ifcase\@last#5\or#3\or#4\or#4\or#4\else#5\fi
    \fi
}
\makeatother

% xassoccnt to make total values: tables, figures, chapters 
%https://tex.stackexchange.com/questions/295857/calculate-amount-of-figures?noredirect=1
\NewTotalDocumentCounter{mytotalfigures}
\NewTotalDocumentCounter{mytotalfiguresInApp}
\NewTotalDocumentCounter{mytotaltables}
\NewTotalDocumentCounter{mytotaltablesInApp}
\NewTotalDocumentCounter{myappendices}
\DeclareAssociatedCounters{figure}{mytotalfigures}
\DeclareAssociatedCounters{table}{mytotaltables}

%https://tex.stackexchange.com/questions/317434/mytotal-pages-number-warning-and-miscalculated
%\NewTotalDocumentCounter{mytotalpages} % not supported yet in xassoccnt, use totpages package
%\DeclareAssociatedCounters{page}{mytotalpages}

%счетчики для вывода на печать
\newcounter{mypages} % счетчик 
\setcounter{mypages}{0} % 
\newcounter{mytotalpagesInApp} % cчетчик 
\setcounter{mytotalpagesInApp}{0} %
\newcounter{myfigures} % счетчик 
\setcounter{myfigures}{0} % 
\newcounter{mytables} % счетчик 
\setcounter{mytables}{0} %  




\AtBeginDocument{
	%% регистрируем счётчики в системе totcounter
	%% позволяет использовать: 
	%% 1) команду \total{counter} для печати значения
	%% 2) спрягать значения слов с помощью \formbytotal
	\regtotcounter{mypages}      % simple counter
	\regtotcounter{TotPages}     % totpages package
	\regtotcounter{myfigures}      % simple counter
	\regtotcounter{mytotalfigures} % xassoccnt package
	\regtotcounter{mytables}      % simple counter
	\regtotcounter{mytotaltables} % xassoccnt package
	\regtotcounter{myappendices}  % xassoccnt package
}
\newtotcounter{citenum} %счетчик для библиографии из totcount package


\preto\appendix{% когда будет команда \appendix 
	% см. также выше переопределение chapter для appendix
	%% Сохранение сумм: рисунки, таблицы, страницы.
	\setcounter{mytotalpagesInApp}{\value{TotPages}}% 
	% count total values
	\AddAssociatedCounters{figure}{mytotalfiguresInApp}
	\AddAssociatedCounters{table}{mytotaltablesInApp}
	\AddAssociatedCounters{chapter}{myappendices}
	%% Форматирование
	%\renewcommand\thechapter{\arabic{chapter}} % см. переопределение chapter для appendix
	\renewcommand{\appendixname}{Приложение} %
	\renewcommand{\thetable}{П\thechapter.\arabic{table}}
	\renewcommand{\thefigure}{П\thechapter.\arabic{figure}}
	\renewcommand{\theequation}{П\thechapter.\arabic{equation}}
	\renewcommand{\thesection}{П\thechapter.\arabic{section}}
	\renewcommand{\thesubsection}{\thesection.\arabic{subsection}}
	\renewcommand{\thesubsubsection}{\thesubsection.\arabic{subsubsection}}
	\counterwithin{footnote}{chapter} %связанная нумерация глав-сносок
	\renewcommand{\thefootnote}{П\thechapter.\arabic{footnote}}
}


%%% Подсчет сумм: рисунки, таблицы, страницы
%% Вариант 1 (рабочий)
\AtEndDocument{
	\setcounter{myfigures}{\value{mytotalfigures}}%
	\addtocounter{myfigures}{-\value{mytotalfiguresInApp}}%
	\setcounter{mytables}{\value{mytotaltables}}%
	\addtocounter{mytables}{-\value{mytotaltablesInApp}}%
	\setcounter{mypages}{\value{mytotalpagesInApp}}%
%	\addtocounter{mypages}{-\value{mytotalpagesInApp}}%
}
%% Вариант 2 (для отладки)
%% работает только в месте вывода на экран суммы, т.е. в реферате
%\setcounter{myfigures}{\numexpr\TotalValue{mytotalfigures}-\TotalValue{mytotalfiguresInApp}\relax}



%%% Правильная нумерация приложений %%%
%% По ГОСТ 2.105, п. 4.3.8 Приложения обозначают заглавными буквами русского алфавита,
%% начиная с А, за исключением букв Ё, З, Й, О, Ч, Ь, Ы, Ъ.
%% Здесь также переделаны все нумерации русскими буквами.
\ifxetexorluatex
    \makeatletter
    \def\russian@Alph#1{\ifcase#1\or
       А\or Б\or В\or Г\or Д\or Е\or Ж\or
       И\or К\or Л\or М\or Н\or
       П\or Р\or С\or Т\or У\or Ф\or Х\or
       Ц\or Ш\or Щ\or Э\or Ю\or Я\else\xpg@ill@value{#1}{russian@Alph}\fi}
    \def\russian@alph#1{\ifcase#1\or
       а\or б\or в\or г\or д\or е\or ж\or
       и\or к\or л\or м\or н\or
       п\or р\or с\or т\or у\or ф\or х\or
       ц\or ш\or щ\or э\or ю\or я\else\xpg@ill@value{#1}{russian@alph}\fi}
    \makeatother
\else
    \makeatletter
    \if@uni@ode
      \def\russian@Alph#1{\ifcase#1\or
        А\or Б\or В\or Г\or Д\or Е\or Ж\or
        И\or К\or Л\or М\or Н\or
        П\or Р\or С\or Т\or У\or Ф\or Х\or
        Ц\or Ш\or Щ\or Э\or Ю\or Я\else\@ctrerr\fi}
    \else
      \def\russian@Alph#1{\ifcase#1\or
        \CYRA\or\CYRB\or\CYRV\or\CYRG\or\CYRD\or\CYRE\or\CYRZH\or
        \CYRI\or\CYRK\or\CYRL\or\CYRM\or\CYRN\or
        \CYRP\or\CYRR\or\CYRS\or\CYRT\or\CYRU\or\CYRF\or\CYRH\or
        \CYRC\or\CYRSH\or\CYRSHCH\or\CYREREV\or\CYRYU\or
        \CYRYA\else\@ctrerr\fi}
    \fi
    \if@uni@ode
      \def\russian@alph#1{\ifcase#1\or
        а\or б\or в\or г\or д\or е\or ж\or
        и\or к\or л\or м\or н\or
        п\or р\or с\or т\or у\or ф\or х\or
        ц\or ш\or щ\or э\or ю\or я\else\@ctrerr\fi}
    \else
      \def\russian@alph#1{\ifcase#1\or
        \cyra\or\cyrb\or\cyrv\or\cyrg\or\cyrd\or\cyre\or\cyrzh\or
        \cyri\or\cyrk\or\cyrl\or\cyrm\or\cyrn\or
        \cyrp\or\cyrr\or\cyrs\or\cyrt\or\cyru\or\cyrf\or\cyrh\or
        \cyrc\or\cyrsh\or\cyrshch\or\cyrerev\or\cyryu\or
        \cyrya\else\@ctrerr\fi}
    \fi
    \makeatother
\fi


%%% Алгоритмы %%%

%\usepackage[linesnumbered]{algorithm2e}
\usepackage[linesnumbered,vlined,figure,scleft]{algorithm2e}

%% Glogal params %%
%ruled, tworuled, algoruled --- put lines to wrap the caption plus a line at the bottom (top) - one should not use this together with inline captions!
%vlined 										--- instead of begin...end will be lines
%boxed 											--- make a box
% figure 										--- count as Fig. ...


% Settings of caption       --- if one will use \caption{} option 	instead of inline + environment caption.
%\SetAlgoCaptionSeparator{.}
%\SetAlgorithmName{Algorithm}{} %last arg is the title of listing table


% Settings for lines numbers
\SetNlSkip{0em}							% sets the value of the space between the line numbers and the text, by default 1em.
\SetNlSty{normalsize}{\hphantom{0}}{.}%defines how to print line numbers
%\hspace*{5mm} does not work 
\SetAlgoNlRelativeSize{-1}	% sets the relative size of line numbers. By default, line numbers are two size smaller than algorithm text

% How to ignore line nuber and to wrap
%http://tex.stackexchange.com/questions/153646/algorithm2e-disabling-line-numbers-for-specific-lines
%http://tex.stackexchange.com/questions/86580/how-to-adjust-line-numbers-of-algorithm2e-package
\makeatletter
%\newcommand{\nosemic}{\renewcommand{\@endalgocfline}{\relax}}% Drop semi-colon ;
%\newcommand{\dosemic}{\renewcommand{\@endalgocfline}{\algocf@endline}}% Reinstate semi-colon ;
%\newcommand{\pushline}{\Indp}% Indent
%\newcommand{\popline}{\Indm\dosemic}% Undent
\let\oldnl\nl% Store \nl in \oldnl
\newcommand{\nonl}{\renewcommand{\nl}{\let\nl\oldnl}}% Remove line number for one line
\makeatother


% Settings for vlines 			
%\SetInd{0.3em}{0.5em}			%default and spaces before and after are 0.5em and 1em
%\SetVlineSkip{5em}					% Sets the value of the vertical space after the little horizontal line which closes a block in vlined mode

%% User abbreviations for ASTRA %%
\SetKwInput{KwInput}{Input}
\SetKwInput{KwOutput}{Output}
%% See also %%
%http://tex.stackexchange.com/questions/145114/caption-below-boxed-algorithm2e-when-used-as-a-figure
%http://tex.stackexchange.com/questions/83536/align-comments-in-algorithm-with-package-algorithm2e



